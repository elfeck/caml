\begin{frame}
  \frametitle{Sparse Grids -- Basics}
  \topline
  \vspace{-10px}
  \begin{block}{Hirachial Basis}
    \begin{figure}[!htp]
      \setbeamertemplate{caption}{\raggedright\insertcaption\par}
      \setbeamerfont{caption}{size=\footnotesize}
      \centering
      \includegraphics[width=7.5cm]{images/sparse_hats}
      \vspace{-12px}
      \caption{}
    \end{figure}
  \end{block}
\end{frame}

\begin{frame}
  \frametitle{Sparse Grids -- Basics}
  \topline
  \vspace{-10px}
  \begin{block}{Hirachial basis (vs nodal basis)}
    \begin{itemize}
      \item Grouping gridpoints into levels $l \in \{1,2,3\dots\}$
      \item Basis function by index \textbf{and} level: $\phi_{l,i}(x)$
    \end{itemize}
    \vspace{20px}
    \begin{center}
      $\hat{f}(x) = \sum_{l,i}{\ \alpha_{l,i} \cdot \phi_{l,i}(x)}$
    \end{center}
  \end{block}
\end{frame}

\begin{frame}
  \frametitle{Sparse Grids -- Basics}
  \topline
  \vspace{-10px}
  \begin{block}{Hirachial Basis}
    \begin{figure}[!htp]
      \setbeamertemplate{caption}{\raggedright\insertcaption\par}
      \setbeamerfont{caption}{size=\footnotesize}
      \centering
      \includegraphics[width=7.5cm]{images/sparse_hats}
      \vspace{-12px}
      \caption{}
    \end{figure}
  \end{block}
\end{frame}

\begin{frame}
  \frametitle{Sparse Grids -- Basics}
  \topline
  \vspace{-10px}
  \begin{block}{Hirachial vs. nodal  basis}
    \begin{figure}[!htp]
      \setbeamertemplate{caption}{\raggedright\insertcaption\par}
      \setbeamerfont{caption}{size=\footnotesize}
      \centering
      \includegraphics[width=7.5cm]{images/sparse_together}
      \vspace{-12px}
      \caption{}
    \end{figure}
  \end{block}
\end{frame}


%%% Local Variables:
%%% TeX-master: "slides"
%%% End:
