\begin{abstract}
  Datasets with large size and high-dimensional data pose a challenge even
  with the steadily growing computational power. To tackle large datasets and
  high dimensionality sparse grids are viable. Due to sparseness and spatial
  adaptivity, high-dimensional and difficult functions can be modeled.
  By modifying least squares estimation, sparse grid can be
  applied to classification and regression with great results in artificial
  and real-world data mining scenarios. Results for the checker-board datasets and a difficult regression task confirm that. 
  Although efficient implemenation of 
  sparse grids is challenging, choosing an iterative approach
  sparse grid can make use of modern hardware features like
  parallelization and vectorization.
\end{abstract}
\vspace{6px}
\begin{keywords}
Sparse grids; Data mining; Hierarchical discretization; Curse of
dimensionality
\end{keywords}

%%% Local Variables:
%%% TeX-master: "report"
%%% End:
