%% Page 1
% Machine learning, big data general
% otivation for grid based methods
% Curse of dimensionality
% Motivation for sparse grids
% Remark properties of sparse grids
\section{Introduction}
Large datasets and high dimensional data remain challenging aspects of data
mining. Even with growing computational power, many problems require
specialized algorithms to archive accurate results within the given time and 
cost restraints.

\par

Sparse grids belong to a more general class of \emph{grid--based}
discretization methods. These methods are primarily applied to
problems including a large amount of datapoints and high--dimensional
feature spaces:
\\
Often, algorithms scale quadratic or worse in the number of datapoints and
thus quickly leading to time and cost related issues.
High dimensionality introduces a problem widely known as the \emph{Curse of
Dimensionality}, denoting an exponential dependency between computational
effort and the number of dimensions.

\par

By focusing on gridpoints instead of the datapoints them self, grid--based
methods are able to deal much better with large datasets. Making the grids
\emph{sparse}, combats the curse of dimensionality while losing accuracy.

\par

Grid--based methods are not restricted to data mining but are also well
suited for a number of different areas including PDA \ref{..}, model order
reduction \ref{..} or numerical quadrature \ref{..}.


In this report the sparse grid technique is applied to data mining
investigating mitigation to the previously mentioned issues.
First this report introduces grid discretization \ref{..} and sparse grids in
general as well as related topics like spatial adaptivity.

Then, sparse grids will be applied machine learning
through modified \emph{least square estimation}. To confirm the capabilities
of sparse grids widely applied, difficult,
test--datasets for regression and classification like the
\emph{checker--board} dataset will be used.

Lastly  efficient implementations on modern systems and parallelization for
sparse grids will be examined.

%%% Local Variables:
%%% TeX-master: "report"
%%% End:
