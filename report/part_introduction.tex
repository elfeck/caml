%% Page 1
% Machine learning, big data general
% otivation for grid based methods
% Curse of dimensionality
% Motivation for sparse grids
% Remark properties of sparse grids
\section{Introduction}
Large datasets and high dimensional data remain challenging aspects of data
mining. Even with growing computational power, many problems require
specialized algorithms to archive accurate results within the given time and
cost restraints.

\par

Sparse grids belong to a more general class of \emph{grid--based}
discretization methods. These methods are primarily applied to
tackle scenarios with large amount of data points and high--dimensional
feature space \cite{artbunshort}, posing the following problems:
\\
Often, algorithms scale quadratic or worse with respect to the number of data points and
thus quickly leading to time and cost related issues.
High dimensionality introduces a problem widely known as the \emph{Curse of
Dimensionality}, denoting an exponential dependency between computational
effort and the number of dimensions of the data \cite{artbunshort, disspfl}. \\
%Grid--based methods are used to tackle these problems.
By focusing on grid points instead
of the data points themself grid--based methods allow for better handling of large amounts of data. 
Sparse grids specifically combat the curse of dimensionality and mitigate
the exponential dependency. \\
Note also, that grid-based approaches are not applicable to data mining
exclusively, but are also
suited for a number of different areas including PDEs, model order
reduction \cite{disspeh} or numerical quadrature \cite{artbunlong}.

\par

In this paper the sparse grid technique is applied to data mining,
investigating the mitigation capabilities to the previously mentioned issues.
First, this paper introduces grid discretization and sparse grids in
general as well as related topics like spatial adaptivity. \\
Then, sparse grids will be applied to machine learning
through \emph{least squares estimation}. To confirm the capabilities
of sparse grids, the results of employing difficult
test-datasets for regression and classification (i.e.
checker-board dataset) will be discussed. \\
Lastly the efficient implementation on modern systems and parallelization of
sparse grids will be examined.

%%% Local Variables:
%%% TeX-master: "report"
%%% End:
