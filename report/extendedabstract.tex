\documentclass[a4paper]{scrartcl}
\usepackage[ngerman]{babel}
\usepackage[T1]{fontenc}
\usepackage[utf8]{inputenc}

\title{Data Mining with Sparse Grids}
\subtitle{Seminar Computational Aspects of Machine Learning}
\author{Sebastian Kreisel}
\date{}

\begin{document}

%\maketitle
\thispagestyle{empty}
\begin{center}
  \huge{\textbf{Data Mining with Sparse Grids}} \\
  \vspace{6px}
  \large{Seminar: Computational Aspects of Machine Learning} \\
  \vspace{15px}
  Sebastian Kreisel
\end{center}
\vspace{30px}
Datasets with large size and high-dimensional data pose a challenge even
with the steadily growing computational power.
Data mining algorithms often scale quadratic or worse in the number of
data points and the computational effort usually grows exponentially with
respect to the
dimensionality of the data.
\par
To tackle these problems \emph{discretization methods} can be employed.
Discretizing the feature space (given by the dataset)
allows to handle a large amount of data
by operating on carefully chosen \emph{grid points} instead of the data points.
Making the grid \emph{sparse} mitigates the exponential
dependency between computational effort and dimensionality without
major drawbacks in accuracy.
\par
This paper explores the sparse grid technique in the context of data mining.
First, the core concepts of discretization will be explained by
discussing full grid discretization of functions
using equidistant grid points. However,
this approach exhibits the \emph{Curse of Dimensionality} and is thus not
applicable to high-dimensional problems. To address that, a hierarchical
structured grid is then examined and
\emph{sparse grids} are introduced which are able to deal with high
dimensionality.
\par
Through applying the sparse grids,
it is possible to find a optimal structured
grid for a general function (\emph{a-priori}). Although this might suffice to
accurately discretize well-behaved functions, machine learning
tasks often require additional care. By introducing \emph{spatial adaptivity}
the grid can be modeled to the specific function at hand and the accuracy
improves even in difficult scenarios.
\par
After these basic notions, sparse grids are
applied to the data mining tasks classification and regression.
The commonly used \emph{least squares
estimation} gets review and then modified to conform with the
previously established formulation of sparse grids. In order to examine
the performance of sparse grids, results of artificial and
real-life data mining scenarios are presented.
\par
In the last section the implementation of sparse grids on modern
systems is discussed briefly. Due to the hierarchical structure some
consideration has to go into an computationally efficient
implementation allowing
parallelization and architecture dependent optimizations.
One approach presented, disregards the nested structure of a hierarchical grid
and trades unnecessary computations for better parallelization with good
results.
\par
To summarize, sparse grids are a viable option when confronted with
high dimensionality and a large amount of data. This is confirmed by
difficult test datasets and existing applications with real scenarios. Although
sparse grids are challenging to implement efficiently there exist approaches
exploiting the capabilities of modern hardware.

\end{document}