% Page 5
\section{Implementation of Sparse Grids}
Many real-world problems require a large number of grid points as mentioned in
Sec.~\ref{subsec:test}. Together with an enormous number of data points and
high dimensionality, a lot of raw computational power is required. To keep
the time and cost as low as possible, an efficient implementation of
sparse grids is essential. Modern hardware offers many features to parallelize
and vectorize computations allowing huge speed-ups if made use of.
\par
Due to the hierarchical basis and the tensor-product structure
a recursive approach seems intuitive. However, this makes it
very hard to parallelize because of unpredictable write-patterns to the
memory which introduces the need of costly precautions \cite{disshei}.
Furthermore, the hierarchical structure of the grid induces scattered
read-accesses to the level and index-vectors of grid points leading to
inefficient memory access and possibly unfavorable caching-behavior \cite{disshei}.
\par
Instead, a more iterative oriented approach can be used. With two nested loops
over all grid points and the dimensionality we compute $\alpha_j \phi_j(x^{(i)})$
for each data point in the inner-most loop. Note, that this leads to a massive amount of
unnecessary computations. This is due to the small support of basis functions
with high level $l$ leading to zero-evaluations which contribute nothing to
the final sum over all basis functions. But even though this iterative
approach is worse computationally, it can make excellent use of parallelization
and vectorization. For both data and grid points all data-dependencies are
removed, eliminating critical sections and make full parallelization possible
\cite{disshei}. Further, due to the sequential, linear access of memory
(stored level-vectors, index-vectors, etc.) the required data can be
 prefetched which removes overhead related to memory bandwidth \cite{disshei}.

%%% Local Variables:
%%% TeX-master: "report"
%%% End:
