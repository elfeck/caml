% Page 5 (, 6)
\section{Conclusion}
Many data mining algorithms are based directly on the training data, often
scaling poorly with regards to the number of data points.
Discretization techniques use grid points and in addition to the data
points to address this issue. By applying full-grid discretization, first
with a equidistant base, then with the hierarchical basis,
to $d$-dimensional functions we introduce the
notion of grid-based methods but also showed that a full grid suffers from
there Curse of Dimensionality.
Sparse grid are able to remedy that by drastically reducing the number of
grid points. Through sparse grid it is possible to obtain a optimal grid
for a general $f$. Even though this might suffice for well-behaved functions,
spatial adaptivity is necessary to model more difficult functions. By refining
the grid in specific areas we are able to capture the structure of very steep
and non-linear functions, often exhibited by machine learning scenarios.
After establishing the sparse grids in general we applied the method to
the data mining tasks classification and regression. Modifying least square
estimation led to a system of linear equations in order to find the
coefficients defining the discretized estimator. After confirming the
performance of sparse grid by showing
results of one artificial and one real-world test, efficient
implementation got discussed briefly. Even though a computationally
efficient implementation of sparse grids is able to exploit the hierarchical
structure, an iterative approach performs better on modern hardware by
trading unnecessary computations for
optimal parallelization and vectorization.
\par
To conclude, sparse grids present a very viable approach to tackle large
datasets and high dimensionality. Even though many subtle factors like
boundary treatment, choice of basis function and refinement strategy have to be
tuned, sparse grids can tackle a wide variety of problems and are already 
employed successfully in real-world scenarios.

%%% Local Variables:
%%% TeX-master: "report"
%%% End:
